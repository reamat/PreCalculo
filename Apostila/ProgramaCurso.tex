\section{Programa do curso}

\begin{enumerate}
\item Conjuntos e Aritmética Básica. Ideia intuitiva de conjunto como uma coleção de elementos. Descrição de um conjunto através da enumeração de seus elementos, ou pela especificação de uma propriedade, ou por diagramas de Venn. Subconjuntos; igualdade de conjuntos. Operações entre conjuntos: união, intersecção, complementar de um conjunto, produto cartesiano de conjuntos. Conjuntos numéricos: Naturais, Inteiros, Racionais, Reais (introduzido pela sua representação decimal como dízima periódica ou não-periódica). Interpretação geométrica dos números reais como pontos de uma reta. Noção de módulo de um número real. Exposição dos axiomas de corpo ordenado dos números reais. Intervalo aberto, intervalo fechado e suas representações geométricas na reta real. Potenciação, radiciação e suas propriedades.

 \item Cálculo com Expressões Algébricas. Produtos notáveis; binômio de Newton. Adição, subtração, multiplicação e divisão de expressões algébricas. Fatoração e simplificação de expressões algébricas; expressões algébricas envolvendo raízes. Polinômio do primeiro grau e análise do sinal do polinômio. Polinômio do segundo grau e análise do sinal do polinômio. Algoritmo da divisão de dois polinômios.

 \item Equações e Inequações. Resolução de equações envolvendo expressões algébricas. Resolução de equações envolvendo expressões algébricas com raízes. Resolução de equações envolvendo módulo de expressões algébricas. Inequações envolvendo expressões algébricas. Inequações envolvendo expressões algébricas com raízes. Inequações envolvendo módulo de expressões algébricas.

 \item Funções. Definição de função, domínio, contradomínio, imagem, gráfico. Funções reais de valores reais. Exemplos: função afim, função quadrática, função definida por várias sentenças. Operações entre funções: adição, subtração, multiplicação, divisão, multiplicação por escalar e composição. Função par, função ímpar, função periódica, função crescente e função decrescente. Função injetora, sobrejetora e bijetora. Função inversa. Construção de gráficos a partir de operações realizadas sobre o gráfico de uma função. Função modular. Funções exponencial e logarítmica; propriedades, gráfico. Resolver equações envolvendo funções exponencial e logaritmo. Resolver inequações envolvendo funções exponencial e logaritmo. Demonstrar identidades envolvendo funções exponencial e logarítmica. Funções hiperbólicas; propriedades, gráfico. Funções trigonométricas e trigonométricas inversas; propriedades, gráfico. Resolver equações envolvendo funções trigonométricas e trigonométricas inversas. Resolver inequações envolvendo funções trigonométricas e trigonométricas inversas. Demonstrar identidades envolvendo funções trigonométricas e funções trigonométricas inversas. Modelagem de situações usando funções.

\end{enumerate}
